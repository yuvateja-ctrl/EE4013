\documentclass[journal,12pt,twocolumn]{IEEEtran}
\usepackage{mathtools}
\DeclarePairedDelimiter\ceil{\lceil}{\rceil}
\DeclarePairedDelimiter\floor{\lfloor}{\rfloor}

\usepackage{forest}
\usepackage{setspace}
\usepackage{gensymb}
\singlespacing


\usepackage{amsthm}

\usepackage{mathrsfs}
\usepackage{txfonts}
\usepackage{stfloats}
\usepackage{bm}
\usepackage{cite}
\usepackage{cases}
\usepackage{subfig}

\usepackage{longtable}
\usepackage{multirow}

\usepackage{enumitem}
\usepackage{mathtools}
\usepackage{steinmetz}
\usepackage{tikz}
\usepackage{circuitikz}
\usepackage{verbatim}
\usepackage{tfrupee}
\usepackage[breaklinks=true]{hyperref}
\usepackage{graphicx}
\usepackage{tkz-euclide}

\usetikzlibrary{calc,math}
\usepackage{listings}
    \usepackage{color}                                            %%
    \usepackage{array}                                            %%
    \usepackage{longtable}                                        %%
    \usepackage{calc}                                             %%
    \usepackage{multirow}                                         %%
    \usepackage{hhline}                                           %%
    \usepackage{ifthen}                                           %%
    \usepackage{lscape}     
\usepackage{multicol}
\usepackage{chngcntr}

\DeclareMathOperator*{\Res}{Res}

\renewcommand\thesection{\arabic{section}}
\renewcommand\thesubsection{\thesection.\arabic{subsection}}
\renewcommand\thesubsubsection{\thesubsection.\arabic{subsubsection}}

\renewcommand\thesectiondis{\arabic{section}}
\renewcommand\thesubsectiondis{\thesectiondis.\arabic{subsection}}
\renewcommand\thesubsubsectiondis{\thesubsectiondis.\arabic{subsubsection}}


\hyphenation{op-tical net-works semi-conduc-tor}
\def\inputGnumericTable{}                                 %%

\lstset{
%language=C,
frame=single, 
breaklines=true,
columns=fullflexible
}
\begin{document}


\newtheorem{theorem}{Theorem}[section]
\newtheorem{problem}{Problem}
\newtheorem{proposition}{Proposition}[section]
\newtheorem{lemma}{Lemma}[section]
\newtheorem{corollary}[theorem]{Corollary}
\newtheorem{example}{Example}[section]
\newtheorem{definition}[problem]{Definition}

\newcommand{\BEQA}{\begin{eqnarray}}
\newcommand{\EEQA}{\end{eqnarray}}
\newcommand{\define}{\stackrel{\triangle}{=}}
\bibliographystyle{IEEEtran}
\raggedbottom
\setlength{\parindent}{0pt}
\providecommand{\mbf}{\mathbf}
\providecommand{\pr}[1]{\ensuremath{\Pr\left(#1\right)}}
\providecommand{\qfunc}[1]{\ensuremath{Q\left(#1\right)}}
\providecommand{\sbrak}[1]{\ensuremath{{}\left[#1\right]}}
\providecommand{\lsbrak}[1]{\ensuremath{{}\left[#1\right.}}
\providecommand{\rsbrak}[1]{\ensuremath{{}\left.#1\right]}}
\providecommand{\brak}[1]{\ensuremath{\left(#1\right)}}
\providecommand{\lbrak}[1]{\ensuremath{\left(#1\right.}}
\providecommand{\rbrak}[1]{\ensuremath{\left.#1\right)}}
\providecommand{\cbrak}[1]{\ensuremath{\left\{#1\right\}}}
\providecommand{\lcbrak}[1]{\ensuremath{\left\{#1\right.}}
\providecommand{\rcbrak}[1]{\ensuremath{\left.#1\right\}}}
\theoremstyle{remark}
\newtheorem{rem}{Remark}
\newcommand{\sgn}{\mathop{\mathrm{sgn}}}
% \providecommand{\abs}[1]{\left\vert#1\right\vert}
% \providecommand{\res}[1]{\Res\displaylimits_{#1}} 
% \providecommand{\norm}[1]{\left\lVert#1\right\rVert}
% %\providecommand{\norm}[1]{\lVert#1\rVert}
% \providecommand{\mtx}[1]{\mathbf{#1}}
% \providecommand{\mean}[1]{E\left[ #1 \right]}
\providecommand{\fourier}{\overset{\mathcal{F}}{ \rightleftharpoons}}
%\providecommand{\hilbert}{\overset{\mathcal{H}}{ \rightleftharpoons}}
\providecommand{\system}{\overset{\mathcal{H}}{ \longleftrightarrow}}
	%\newcommand{\solution}[2]{\textbf{Solution:}{#1}}
\newcommand{\solution}{\noindent \textbf{Solution: }}
\newcommand{\cosec}{\,\text{cosec}\,}
\providecommand{\dec}[2]{\ensuremath{\overset{#1}{\underset{#2}{\gtrless}}}}
\newcommand{\myvec}[1]{\ensuremath{\begin{pmatrix}#1\end{pmatrix}}}
\newcommand{\mydet}[1]{\ensuremath{\begin{vmatrix}#1\end{vmatrix}}}
\numberwithin{equation}{subsection}
\makeatletter
\@addtoreset{figure}{problem}
\makeatother
\let\StandardTheFigure\thefigure
\let\vec\mathbf
\renewcommand{\thefigure}{\theproblem}
\def\putbox#1#2#3{\makebox[0in][l]{\makebox[#1][l]{}\raisebox{\baselineskip}[0in][0in]{\raisebox{#2}[0in][0in]{#3}}}}
     \def\rightbox#1{\makebox[0in][r]{#1}}
     \def\centbox#1{\makebox[0in]{#1}}
     \def\topbox#1{\raisebox{-\baselineskip}[0in][0in]{#1}}
     \def\midbox#1{\raisebox{-0.5\baselineskip}[0in][0in]{#1}}
\vspace{3cm}
\title{Assignment 1}
\author{Yuvateja - EE18BTECH11043}
\maketitle
\newpage
\bigskip
\renewcommand{\thefigure}{\theenumi}
\renewcommand{\thetable}{\theenumi}
Download all latex-tikz codes from 
%
\begin{lstlisting}
https://github.com/yuvateja-ctrl/EE4013/blob/main/assignment1.tex
\end{lstlisting}
\section{Problem}
(Q 24) Consider the following representation of a number in IEEE-754 single precision floating point format with a bias of 127

S : 1,   E : 10000001 ,  F : 11110000000000000000000

Here S, E and F denote the sign, exponent and fraction components of the floating point representation.

The decimal value corresponding to the representation ( rounded to 2 decimal places ) is 


\begin{lstlisting}
#include<stdio.h>
#include<stdlib.h>
#include<math.h>

int main()
{
    char S[] = "1";
    char E[] = "10000001";
    char M[] = "11110000000000000000000";

    int e = sizeof(E)/sizeof(E[0]);
    int m = sizeof(M)/sizeof(M[0]);

    printf("%d .. %d\n",e,m);

    int exponent = 0;
    for(int i = 0;i<e-1;i++)
    {
        if(E[i] == '1')
        {
            exponent = exponent*2 + 1;
        }
        else
        {
            exponent = exponent*2 + 0;
        }
        printf("%d\n",exponent);
    }
    
    double fraction = 0;
    for(int i = 0;i<m-1;i++)
    {
        if(M[i] == '1')
        {
            fraction += 1*(pow(2,-i-1));
        }
    }
    int s = (S[0] == '1')?-1:1;
    printf("%lf\n",fraction);
    double ans = s*(1+fraction)*(pow(2,exponent-127));
    printf("The decimal value corresponding to above representation is %lf \n",ans);
    

    return 0;
}
\end{lstlisting}
The output of the program upon execution is 
\section{Solution}
Answer : -7.74
\newline
\newline
\textbf{Explanation}
\newline
\newline

In IEEE-754 single precision format a floating point number is represented in 32 bits.
\newline
\newline
1.Sign bit( MSB ) - 1 bit
\newline

2.biased exponent ( E' ) - 8 bits
\newline

3.Normalized mantissa ( M ) - 23 bits
\newline

Sign bit value 0 means positive number, 1 means negative number.
\newline
\newline
The floating point number can be obtained by formula \pm 1.M * 2 ** (E - 127)
\newline
\newline
\nextpage

Given sign bit is 1 , i.e, the number is negative
\newline
\newline
Biased exponent (E')  is  10000001 =  128 + 1 = 129
\newline

Normalized Mantissa (M) is 11110000000000000000000  =  0.937 
\newline
\newline

Therefore the decimal value representation is calculated by above mentioned formula  
\newline
\newline
\pm 1.M * 2 ** (E - 127)  = -1.937 * ( 2 ** ( 129 - 127 )) =  -1.937 * (2**2) = -7.748 = -7.75

( rounded to 2 decimals ) 

  
\end{document}
\end{document} 
